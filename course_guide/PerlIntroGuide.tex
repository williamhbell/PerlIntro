\documentclass[17pt,dvips]{foils}
\usepackage{graphics}
\usepackage{latexsym}
 %
 % Declare the title, author and date as you would in regular \LaTeX.
 %
\title{An introduction to Perl}
 %
\author{W. H. Bell\\
       \texttt{W.Bell@cern.ch}
       }
 % This is optional
\date{\copyright 2010}

\MyLogo{Perl shell programming}   % this is the default
\rightfooter{\textsf{\thepage}}   % this is the default \thepage
\leftheader{W. H. Bell}
\rightheader{2010}                % \filedate is defined above
 %
\newcommand\bs{\char '134}   %  a backslash character for the \tt font
 %
 % Now we can begin the document.  The first thing is the title page
 % on which we might put a VERY short abstract.
% Extra commands
\def\perl{\texttt{perl}$\;$}

\begin{document}

\maketitle
 %
\begin{abstract}
\begin{center}
A basic overview of the \perl scripting language is given.  The
language is introduced using examples covering a few aspects at a
time.  Reference tables are provided for commonly used \perl
functionality.
\end{center}
\end{abstract}

%%%%%%%%%%%%%%%%%%%%%%%%%%%%%%%%%%%%%%%%%%%%%%%%%%%%%%%%%%%%%%%%%%%%%%%%%%%%

\foilhead{Perl - Introduction}
\noindent There are lots of manual pages:
\begin{verbatim}
[wbell@ppepc54 ~]$ man perl
\end{verbatim}
%$
For ease of access, the Perl manual has been split up into
several sections.  (The way the manual page sections is divided
depends on the version of Perl)

\noindent For the less man page adept there is also a web site:
http://www.perl.com/pub/q/documentation which presents the same
information.

% Need a quick word about man pages.  (Man normally uses less i.e. can
% search as one does in less.)

%%%%%%%%%%%%%%%%%%%%%%%%%%%%%%%%%%%%%%%%%%%%%%%%%%%%%%%%%%%%%%%%%%%%%%%%%%%%

\foilhead{Perl - Overview}
\begin{itemize}
\item A basic program
\item Syntax
  \begin{itemize}
  \item Loops
  \item Conditional Statements
  \item Functions
  \end{itemize}
\item Variable Types
% To be covered inside other examples.
  \begin{itemize}
  \item Scalars
  \item Arrays
%  \item Hashes
% Just mention hashes at this point.
  % Mention globs in passing as pointer like objects.
  \end{itemize}
\item Files Handles and their Usage
  \begin{itemize}
  \item Reading and Writing
  \item Using Commands
  \end{itemize}
\item Command Line Arguments
\item String Manipulation
  \begin{itemize}
  \item Pattern Matching
  \item Subsitution
  \item Splitting and Joining
  \end{itemize}
\end{itemize}

%%%%%%%%%%%%%%%%%%%%%%%%%%%%%%%%%%%%%%%%%%%%%%%%%%%%%%%%%%%%%%%%%%%%%%%%%%%%

\foilhead{Perl - A Basic Program}
\begin{verbatim}
#!/usr/bin/perl

print STDOUT "In the beginning\n";
\end{verbatim}
{\bf example\_01.pl}: A basic perl program

\begin{itemize}
\item The first line: \texttt{PATH} to executable.
\item The second line: print via the standard out.
\end{itemize}

% The first line is the same for any shell script.  It is the PATH to 
% the executable that will interpret the script.  Note if this path is 
% wrong it will report command not found.  One should use 'which' to 
% check the location of perl.

%%%%%%%%%%%%%%%%%%%%%%%%%%%%%%%%%%%%%%%%%%%%%%%%%%%%%%%%%%%%%%%%%%%%%%%%%%%%

\foilhead{For Loops}

\begin{verbatim}
# for loop
for ($i=1; $i<=2; $i++) {
  print STDERR "for loop $i\n";
}

# for loop 2
@fruits = ("apple", "pear", "plumb");

for my $fruit (@fruits) {
  print STDOUT "$fruit ";
}
print STDOUT "\n";

# for loop 3
for (1..2) {
  print STDOUT "Loop $_\n";
}
\end{verbatim}
%$
{\bf example\_02.pl}: A perl program demonstrating different types of loops.

% Chat briefly about arrays and their usage.

\begin{itemize}
\item The C style is slower than the foreach style of for loop.
\item Try to code in the Perl paradigm
\end{itemize}

% The 'for each' style of the second loop is faster than the former
% example.  Should program in the latter style if possible.
%============================================================

\foilhead{More Loops}

\noindent \texttt{while (}{\em condition}\texttt{) \{}\\
{\em statement}\texttt{;\\
\}}

\noindent \texttt{do \{}\\
{\em statement}\texttt{;\\
\} until} {\em condition}\texttt{;}

\noindent \texttt{do \{}\\
{\em statement}\texttt{;\\
\} while (}{\em condition}\texttt{);}

\begin{itemize}
\item The while loop is implemented within {\bf example\_02.pl}.
\end{itemize}

%============================================================

\foilhead{Conditional Statements: if}
\begin{verbatim}
#!/usr/bin/perl

for (1..3) {
  if($_==1) {
    print STDOUT "1, ";
  }
  elsif($_==2) {
    print STDOUT "2, ";
  }
  else {
   print STDOUT "3!\n";
  }
}
\end{verbatim}
%$
{\bf example\_03.pl}: A program demonstrating an \texttt{if}, \texttt{elsif}, 
\texttt{else} statement.

\begin{itemize}
\item Note the usage of the \texttt{\$\_} variable.
\item If the output of an argument is not assigned to a variable it is
implicitly assigned to \texttt{\$\_}.
\end{itemize}

%============================================================

\foilhead{Conditional Statements: die, unless}
\begin{verbatim}
#!/usr/bin/perl

for (1..5) {
  die "It's time to die." unless ($_<4);
  print STDOUT "$_\n";
}
\end{verbatim}
%$
{\bf example\_04.pl}: A simple program to illustrate the use of die, unless
conditional statements.

\begin{itemize}
\item The \texttt{die} message is printed together with a line number.
\item Useful for critical checks, e.g. opening files etc.
\end{itemize}

%============================================================

\foilhead{Functions}
\begin{verbatim}
#!/usr/bin/perl

printStrings(1,2,3);

sub printStrings {
  my ($a, $b, $c) = @_;
  
  print STDOUT "a=$a, b=$b, c=$c\n";
}
\end{verbatim}
%$
{\bf example\_05.pl}: A simple script to demonstrate basic function usage.

\begin{itemize}
\item The usage of \texttt{@\_} is similar to that of \texttt{\$\_}.
The values are assigned implicitly.
\item Warning: passing arrays in this way can be tricky.
\end{itemize}


%============================================================

\foilhead{File I/O - Writing}

\begin{verbatim}
$outputfile = "perl_output";

die ("Can't open $outputfile for writing\n")
  unless open(OUTPUT, "> $outputfile");

print OUTPUT "Pythia 6.2\n";
print OUTPUT "EvtGen On\n";
print OUTPUT "Geant v4\n";
   
close(OUTPUT);
\end{verbatim}
%$
{\bf example\_06.pl}: A program that illustrates usage of file i/o. [The
writing section.]

\begin{itemize}
\item The file handle works in the same way as the STDOUT handle.
\end{itemize}

%============================================================

\foilhead{File I/O - Reading}

\begin{verbatim}
die ("Can't open $outputfile for reading\n")
  unless open(INPUT, "< $outputfile");

while (<INPUT>) {
  chomp;
  if(defined $_) {
    print STDOUT "$_\n";
  }
}
close(INPUT);

unlink($outputfile);
\end{verbatim}
%$
{\bf example\_06.pl}: A program that illustrates usage of file i/o. [The
reading section.]

% Note the usage of $_ and chomp.

\begin{itemize}
\item The value following the file handle can be many different
things. It behaves like a commnand line, i.e. pipe and redirection can
be used.
%\item All unix shell commands are available via this second argument.
\end{itemize}

%============================================================

\foilhead{Array Manipulation}
\begin{verbatim}
... 
my @run_numbers;
print STDOUT "\@run_numbers last index=";
while ($run = <RUNFILE>) {
  chomp($run);
  push(@run_numbers,$run);
  print STDOUT "$#run_numbers... ";
}
print STDOUT "\n\n";
 
print STDOUT "Run Numbers\n";
while ($#run_numbers >= 0) {
  $run=shift(@run_numbers);
  print STDOUT "$run\n";
}
print STDOUT "\n";
                                                                               print STDOUT "\@run_numbers size=$#run_numbers\n";
\end{verbatim}
%$
{\bf example\_07.pl}: A program to demonstrate dynamic array management

\begin{itemize}
\item \texttt{push} and \texttt{pop}, add and remove from the end of the array.
\item \texttt{unshift} and \texttt{shift}, add and remove from the beginning of the array.
\end{itemize}

%============================================================

\foilhead{Using Commands}

\begin{verbatim}
# find files which have been modified in the last 60 minutes and open
# the output as a file input stream.
open(FINDFILE,  "find    ./ -cmin -60 |");

# While there are files returned from the find command
while (<FINDFILE>) {
  chomp;
  if(defined $_) {
     print STDOUT " $_ was in the last 60 minutes.\n";
  }
}
close(FINDFILE);
\end{verbatim}
%$
{\bf example\_08.pl}: A program to demonstrate the usage commands piped into a
file input stream

% Note chomp and the usage of $_.

\begin{itemize}
\item Any executable can be run in a similar manner.
\end{itemize}

%============================================================

\foilhead{Command Line Arguments}

\begin{verbatim}
$numArgs = $#ARGV + 1;
print STDOUT "\n $0 with $numArgs command-line arguments\n\n ";

foreach $argnum (0 .. $#ARGV) {
   print STDOUT "\$ARGV[$argnum] = $ARGV[$argnum], ";
}

print STDOUT "\n\n";
\end{verbatim}
%$
{\bf example\_09.pl}: A program demonstrating usage of command line arguments

\begin{itemize}
\item Arguments are accessed via \texttt{@ARGV}.
\item \texttt{\$\#ARGV} index of last argument in array.
\item Executable name via \texttt{\$0}
\end{itemize}

%============================================================

\foilhead{Pattern Matching}

\begin{verbatim}
open(FILELIST,  "ls |");

while ($file = <FILELIST>) {
  chomp($file);
  if(defined $file) {
    if($file =~ /ex[1-4]\.pl/) {
      print STDOUT " $file \n";
    }
  }
}
close(FILELIST);
\end{verbatim}
%$
{\bf example\_10.pl}: A script to demonstrate pattern matching with regular
expressions.

% Mention the usage of the file handle without $_ and the =~ syntax.

\begin{itemize}
\item Matches file containing \texttt{ex} followed by one of 1,2,3,4
and \texttt{.pl}.  (If the file was called fedex1.pl it would match
too.)
% Need to use an anchor to avoid this problem.
\end{itemize}

%============================================================

\foilhead{Pattern Matching}

\begin{verbatim}
open(FILELIST,  "ls |");

while ($file = <FILELIST>) {
  chomp($file);
  if(defined $file) {
    if($file =~ /^ex\d.[\w]/) {
      print STDOUT " $file \n";
    }
  }
}
close(FILELIST);
\end{verbatim}
%$
{\bf example\_11.pl}: A program to demonstrate using an anchor and decimal and word

\begin{itemize}
\item The program only selects files beginning with \texttt{ex}
followed by a single number character and then any word extension
e.g. \texttt{pl} or \texttt{sh}.
\end{itemize}

%============================================================

\foilhead{Substitution}

\begin{verbatim}
@al = ('a','b','c','d','e','f','g','h','i','j','k','l','m','n',
       'o','p','q','r','s','t','u','v','w','x','y','z');

@AL = ('A','B','C','D','E','F','G','H','I','J','K','L','M','N',
       'O','P','Q','R','S','T','U','V','W','X','Y','Z');
     
open(LISTOF,"ls |");

while (<LISTOF>) {
  chomp;
  if(defined $_) {
    $i = 0;
    while ($i<26) {
      s/$al[$i]/$AL[$i]/g;
      $i = $i + 1;
    }
    print STDOUT "$_\n";
  }
}
close(LISTOF);
\end{verbatim}
%$
{\bf example\_12.pl}: A program to demonstrate substitution.

% global is needed to replace all instances.  If it isn't used only
% the first instance will be replaced.

%============================================================

\foilhead{Splitting and Joining}

\begin{verbatim}
while (<TABFILE>) {
  chomp;
  if(defined $_) {
    @data = split(/\t/,$_);
    $data_line = join(',',@data);
    print CVSFILE "$data_line\n";
  }
}

close(TABFILE)
\end{verbatim}
%$
{\bf example\_13.pl}: A program to demonstrate the split and join.


\end{document}
\endinput
